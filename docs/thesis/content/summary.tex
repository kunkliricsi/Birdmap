%----------------------------------------------------------------------------
\chapter{Értékelés}
\label{chapt:summary}
%----------------------------------------------------------------------------
Úgy gondolom, hogy az alkalmazásom elérte a célját.
Egy használható felületet nyújt a Birbnetes mikroszolgáltatás rendszere működésének vizualizálására.
A fejlesztés közben jelentős figyelmet fordítottam arra, hogy az alkalmazás felületi és kód komponensei között is
minimalizáltak legyenek a függőségek, így a rendszerben történő változások esetén azok könnyen cseréhetőek, bővíthetőek.
%----------------------------------------------------------------------------
\section{Továbbfejlesztési lehetőségek}
%----------------------------------------------------------------------------
Az kliens oldalon történő diagramok adatainak generálása hamar túl nagy falatnak bizonyult.
A bevetett optimalizációk ellenére sem lett hatványozottan gyorsabb a felület.
Így az első és legfontosabb továbbfejlesztési teendő az adatok szerveroldalon történő generálása lenne.

A Logs oldal jelenleg csak a szerveroldalon készült napló fájlokat tartalmazza.
Hasznos lenne, ha az egyes mikroszolgáltatások naplófájljai is letölthetőek lennének.

Ezen kívül előnyös lenne a rendszer belső működését vizualizáló komponensek alkalmazása is, 
ahol lehetne látni az egyes mikroszoltáltatásokra vonatkozó különböző metrikákat például az adatfeldolgozási időt vagy a beérkezett kérések számát. 