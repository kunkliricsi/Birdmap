%----------------------------------------------------------------------------
\chapter{\bevezetes}
%----------------------------------------------------------------------------
Szőlőtulajdonosoknak éves szinten jelentős kárt okoznak a seregélyek, akik előszeretettel választják táplálékul a megtermelt szőlőt.
Erre a problémára dolgoztak ki a tanszéken diáktársaim egy felhőalapú konténerizált rendszert, a Birbnetes-t
mely a természetben elhelyezett eszközökkel kommunikál, azokat vezérli.
Az eszközök bizonyos időközönként hangfelvételt készítenek a környezetükről,
majd valamilyen formában elküldik ezeket a felvételeket a központi rendszernek,
amely egy erre a célra kifejlesztett mesterséges intelligenciát használva eldönti
a felvételről, hogy azon található-e seregély hang vagy sem.
Ha igen akkor jelez a felvételt küldő eszköznek, hogy szólaltassa meg a riasztó
berendezését, hogy elijessze a madarakat.

%----------------------------------------------------------------------------
\section{Probléma}
%----------------------------------------------------------------------------
A jelen rendszer használata során nincs vizuális visszacsatolás az esetleges riasztásokról azok gyakoriságáról
és a rendszer állapotáról sem. Különböző diagnosztikai eszközök ugyan implementálva lettek, mint például
a naplózás vagy a hiba bejelentés, de ezek használata nehézkes, nem kézenfekvő. 
Szükség van egy olyan megoldásra, amivel egy helyen és egyszerűen lehet kezelni és felügyelni a rendszer egyes elemeit.

%----------------------------------------------------------------------------
\section{Megoldás}
%----------------------------------------------------------------------------
A jelen szakdolgozat egy olyan webes alkalmazás elkészítését dokumentálja, mellyel a felhasználók képesek
a természetben elhelyezett eszközök állapotát vizsgálni, azokat akár ki és bekapcsolni igény szerint.
Az egyes rendszer eseményeket vizsgálva a szoftver statisztikákat készít, melyeket különböző diagramokon ábrázolok.
Ilyen statisztikák például, hogy időben melyik eszköz mikor észlelt madárhangot, vagy hogy hány hang üzenet érkezik
az eszközöktől másodpercenként.

%----------------------------------------------------------------------------
\section{A szakdolgozat felépítése}
%----------------------------------------------------------------------------
A szakdolgozatom első részében, a \ref{chapt:birdnetes-introduction}. fejezetben, bemutatom a vizualizálni kívánt rendszer felépítését, az egyes komponensek közötti kapcsolatokat,
valamint a vizualizációs szempontból releváns technológiákat, amire a rendszer épült.
A \ref{chapt:birdmap-introduction} fejezetben ismertetem a jelenleg az iparban is használt mikroszolgáltatás működését vizualizáló alternatívákat, majd a saját megoldásom tervezetét, az arra vonatkozó elvárásokat.
A \ref{chapt:birdmap-technologies} fejezetben az alkalmazásom által használt technológiákat mutatom be, 
ezzel előkészítve az \ref{chapt:birdmap-backend} és \ref{chapt:birdmap-frontend} fejezetet, ahol ismertetem a szerver- és kliensalkalmazások felépítését.
A \ref{chapt:birdmap-test} és \ref{chapt:birdnetes-kubernetes} fejezet az alkalmazás teszteléséről és telepítéséről szól.
Az utolsó fejezetben értékelem a munkám eredményét, levonom a tapasztalatokat és bemutatok néhány továbbfejlesztési lehetőséget.