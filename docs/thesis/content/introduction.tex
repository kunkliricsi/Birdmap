%----------------------------------------------------------------------------
\chapter{\bevezetes}
%----------------------------------------------------------------------------
A tanszéken diáktársaim készítettek egy felhő alapú konténerizált rendszert, a Birdnetes-t
mely a természetben elkelyezett eszközökkel kommunikál, azokat vezérli.
Az eszközök bizonyos időközönként hangfelvételt készítenek a környezetükről,
majd valamilyen formában elküldik ezeket a felvételeket a központi rendszernek,
amely egy erre a célra kifejlesztett mesterséges intelligenciát használva eldönti
a felvételről, hogy azon található-e madár, konkrétan seregély hang vagy sem.
Ha igen akkor jelez a felvételt küldő eszköznek, hogy szólaltassa meg a riasztó
rendszerét, hogy elijessze a seregélyt.

%----------------------------------------------------------------------------
\section{A probléma}
%----------------------------------------------------------------------------
A jelen rendszer használata során nincs vizuális visszacsatolás az esetleges riasztásokról azok gyakoriságáról
és a rendszer állapotáról sem. Különböző diagnosztikai eszközök ugyan implementálva lettek mint például
a logolás vagy a hiba bejelentés, de ezek használata nehézkes, nem kézenfekvő. 
Szükség van valamire amivel egy helyen és egyszerűen lehet kezelni és használni a rendszer egyes elemeit.

%----------------------------------------------------------------------------
\section{A megoldás}
%----------------------------------------------------------------------------
A jelen szakdolgozat egy olyan webes rendszer elkészítését dokumentálja, melyel a felhasználók képesek
a természetben elhelyezett eszközök állapotát vizsgálni, azokat akár ki és bekapcsolni igény szerint.
Az egyes rendszer eseményeket vizsgálva a szoftver statisztikákat készít, melyeket különböző diagrammokon ábrázolok.
Ilyen statisztikák például, hogy időben melyik eszköz mikor észlelt madár hangot, vagy hogy hány hang üzenet érkezik
az eszközöktől másodpercenként.


%----------------------------------------------------------------------------
\section{A szakdolgozat felépítése}
%----------------------------------------------------------------------------
A szakdolgozatom első részében bemutatom a Birdnetes 