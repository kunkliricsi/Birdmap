%----------------------------------------------------------------------------
\chapter{Docker image készítés}
\label{chapt:birdnetes-kubernetes}
%----------------------------------------------------------------------------
Az éles rendszerrel való kommunikáció megvalósításához készítenem kell egy Docker image-et, melyet telepíteni lehet a Birbnetes Kubernetes klaszterébe.
Ehhez először készítettem egy Dockerfile-t \cite{dockerfile}, mely az image-ek automatikus elkészítését teszi lehetővé.
Utasításokat lehet benne felsorolni, melyekkel a konténer környezetét kell felépíteni.
Meg lehet adni kiindulópontokat, mely az image alapjául szolgál.
Erre a célra én az ASP.NET futtatokörnyeztét használtam, mely tartalmazza az alkalmazás futtatásához szükséges parancsokat.
Ezek után a Dockerfile utasításait használva bemásolom a \verb+Release+ konfigurációval fordított alkalmazásomat a konténer egy mappájába, 
majd a belépési pont utasítással megadom az alkalmazás indításához szükséges parancsot.
Ezt futtatva sikeresen elkészül a Docker image.

Azonban az alkalmazás teljes értékű működéséhez annak szüksége van egy adatbázis konténerre is.
Az ilyen jellegű többkonténeres rendszer problémákra nyújt megoldást a Docker Compose \cite{docker-compose}.
Egy YAML fájlban meg lehet adni az alkalmazás futtatásához szükséges szolgáltatásokat, illetve hogy ezek között milyen függőségi viszony van.
Ennek használatával először készítek egy adatbázis konténert, mely inicializálása után indul csak el az alkalmazásom docker image-ének készítése.
A két konténer közötti kommunikációhoz az alkalmazásomnak szüksége van még a kapcsolati karakterláncra, mely meghatározza az adatbázis elérésének paramétereit.
A lokális futtatásnál ez az alkalmazás konfigurációs fájljában található, azonban ez a fájl már a konténer fájlrendszerében van, nehézkes hozzáférni.
Szerencsére az ASP.NET támogatja a konfigurációk felülírását környezeti változókkal.
Ehhez fel kell sorolnom a YAML fájl környezeti változói részében a felülírni kívánt konfigurációkat és értékeiket.
Szintén ebben a fájlban megadtam az alkalmazás eléréséhez használni kívánt portokat.
Ezek után az alkalmazásom készen áll a klaszterbe való telepítésre.