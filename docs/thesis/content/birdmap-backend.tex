%----------------------------------------------------------------------------
\chapter{Backend}
\label{chapt:birdmap-backend}
%----------------------------------------------------------------------------
Ebben a fejezetben bemutatom a szerveroldal architektúráját, felépítését. Ismertetem a különböző szoftver komponensek feladatát.

%----------------------------------------------------------------------------
\section{Architektúra}
%----------------------------------------------------------------------------
A szerveroldal fejlesztésénél a háromrétegú architektúrát alkalmaztam, melynek lényege, hogy az alkalmazást logikailag három elkülönülő részre bontjuk:
\begin{itemize}
    \item \textbf{Adat elérési réteg}. Ez a rész felel a tárolt entitások modell definícióiért, illetve azoknak a kiolvasásáért, tárolásáért egy adatbázisból vagy fájlrendszerből.
    \item \textbf{Megjelenítési réteg}. Ezen réteg feladata a kliensoldal közvetlek kiszolgálása. Bármilyen irányú kommunikáció a kliensek felé ezen a rétegen keresztül történik.
    \item \textbf{Üzleti logikai réteg}. Minden ami nem a közvetlen kommunikációért, megjelenítésért vagy adat elérésért, tárolásért felel, az ide kerül.
    A fenti két réteg között helyezkedik el és feladata a különböző folyamatok értékelése és futtatása, valamint az adatok feldolgozása.
\end{itemize}

\lstset{language=C,keywordstyle={\bfseries \color{blue}}}
Az ASP.NET Core beépítetten támogatja a dependency injection-t, mely a \verb+Startup+ osztály \verb+ConfigureServices+ metódusával konfigurálható.
Én minden rétegbe tettem egy ilyen \verb+Startup+ osztályt, hogy azok feleljenek a saját szolgáltatásaik konfigurálásáért és regisztrálásáért.

%----------------------------------------------------------------------------
\section{Adat elérési réteg}
%----------------------------------------------------------------------------
Az adatelérést az Entity Framework Core segítségével oldottam meg. Telepítettem egy MSSQL adatbázis szervert a számítógépemre, melynek csatlakozási paramétereivel
a \verb+Startup+ osztályban felkonfigurálom az EF Core által nyújtott \verb+DbContext+ saját leszármazott változatát. 
Így csak az entitások elkészítése és azok alapértelmezett értékeinek az adatbázisba való feltöltése marad hátra.

%----------------------------------------------------------------------------
\subsection{Entitások}
%----------------------------------------------------------------------------
Mivel az adatok nagy részét külső szolgáltatások fogják nyújtani, így lokálisan összesen két entitás létrehozására volt szükség. 
Az egyik a \verb+User+, mely az alkalmazás felhasználóinak adatait tárolja. 
A másik a \verb+Service+, mely a külső szolgáltatások adatainak tárolását szolgálja, amelyeket azért tárolok az adatbázisban és nem mondjuk a konfigurációs fájlban,
mert szerettem volna, hogyha a kezelőfelületen lehetne őket szerkeszteni, törölni.

\lstset{style=sharpc, morekeywords={record, get, set}}
\begin{lstlisting}[caption=A User és a Service modell]
    public record User
    {
        public int Id { get; set; }
        public string Name { get; set; }
        public byte[] PasswordHash { get; set; }
        public byte[] PasswordSalt { get; set; }

        public Roles Role { get; set; }

        public bool IsFromConfig { get; set; }
    }

    public record Service
    {
        public int Id { get; set; }
        public string Name { get; set; }
        public Uri Uri { get; set; }

        public bool IsFromConfig { get; set; }
    }
\end{lstlisting}

Az alkalmazás használata szempontjából a felhasználók két csoportba oszlanak.
Vannak adminisztrátor és sima felhasználók, utóbbi csak az adatok olvasására, míg előbb azok módosítására is jogosult.
Ennek a megkülönböztetésnek a jelzője a \verb+Role+ osztály tulajdonság.

%----------------------------------------------------------------------------
\subsection{Seedelés}
%----------------------------------------------------------------------------
Az alkalmazás konfigurációs fájljából meg lehet adni alapértelmezett felhasználókat és szolgáltatásokat.
Ezeknek megkülönböztetésére szolgál az entitások \verb+IsFromConfig+ mezője.
A szerver indítása legelején, megvizsgálja, hogy létezik-e az adatbázis és ha igen kitöröl minden olyan entitást ahol az \verb+IsFromConfig+ mező igaz.
Majd hozzáadja az újonnan beolvasott értékeket.

%----------------------------------------------------------------------------
\section{Üzleti logikai réteg}
%----------------------------------------------------------------------------
Ebben a rétegben található meg a szerver legtöbb szolgáltatása. It vannak implementálva a Birdnetes Command and Control és Input komponenseivel kommunikáló szolgáltatások is, 
melyeket azok OpenAPI leírói alapján az NSwag\cite{nswag} alkalmazással generáltam.

Ugyan itt található meg a \verb+User+ és \verb+Service+ entitások létrehozásáért, olvasásáért, szerkesztéséért és törléséért felelős szolgáltatások is.
Valamint itt található még az autentikációért felelős szolgáltatás is. A felhasználók jelszavainak tárolására a HMAC (Hash-based Message Authentication Code) algorithmust, 
pontosabban annak a \verb+HMACSHA512+ C\# implementációját használtam.

Minden jelszóhoz generálok egy egyedi kulcsot és azzal egy hash-t és ezeket tárolom a \verb+User+ modell \verb+PasswordSalt+ és \verb+PasswordHash+ mezőiben.
Amikor egy felhasználó be akar jelentkezni először megvizsgálom, hogy egyáltalán létezik-e az adatbázisban a megadott nevű felhasználó 
és ha igen a megadott jelszóból az imént említett folyamattal generált kulcsot és hash-t összehasonlítom az adatbázisban tárolttal.

%----------------------------------------------------------------------------
\section{Megjelenítési réteg}
%----------------------------------------------------------------------------
