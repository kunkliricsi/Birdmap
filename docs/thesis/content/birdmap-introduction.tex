%----------------------------------------------------------------------------
\chapter{Tervek és alternatívák}
\label{chapt:birdmap-introduction}
%----------------------------------------------------------------------------
Ebben a fejezetben bemutatom a fejlesztés előtti állapotot, amikor még csak tervezgettük, hogy milyen is legyen az alkalmazás.
Illetve bemutatok, néhány vizualizációs alternatívát, melyek jó iránymutatásként szólgálltak a fejlesztés során.

%----------------------------------------------------------------------------
\section{Tervezés}
%----------------------------------------------------------------------------
Az első dolgom az volt, hogy Kristóffal és Marcellel beültünk egy Teams\footnotemark-en tartott gyűlésre,
ahol elmagyarázták nagyvonalakban, hogy hogyan is működik a rendszer, mik az egyes kompnensek feladatai. 
Ezek után az előttem álló fejlesztésre váró alkalmazás részleteit beszéltük meg, az elvárt igényeket azzal kapcsolatban.
Itt rögtön több ötlet is felmerült, melyek közül a legkiemelkedőbbek:
\begin{itemize}
    \item \textbf{Hőtérkép}. Hasznos lenne egy olyan felület, ahol az eszközök GPS koordinátái és a seregély detektálást jelző üzenetek alapján, meg lehetne jeleníteni a seregélyek hozzávetőleges előfordulásának helyeit és gyakoriságát egy térképen, hőtérképes formában.
    \item \textbf{Eszköz állapotok}. Jelenleg a Command and Control mikroszolgáltatás felé indított kéréseken kívül, nincs lehetőség a kihelyezett eszközök állapotának vizsgálatára. Szükség lenne egy olyan felületre, ahol ezek állapotai láthatóak, esetleg dinamikusan is frissülnek.
    \item \textbf{Diagrammok}. A hőtérképen kívül egyéb olyan diagrammok is hasznosak lehetnek, ahol látható például, hogy melyik eszköz melyik percben észlelt madárhangot vagy, hogy egy eszköz összesen hány madárhangot észelt. Minnél több információ, annál jobb.
\end{itemize}
Ezeken kívül fontos követelmény volt még, hogy az alkalmazásom futtatható legyen Linux környezetben is, hogy az telepíthető legyen a Birdnetes Kubernetes\cite{kubernetes} klaszterébe.

Az alkalmazásom kapott egy nevet is, mely a Birdnetes-t és az említett hőtérképes ötletet ötvözve Birdmap lett.
\footnotetext{Microsoft Teams: Csevegő és gyülekezés tartó alkalmazás.}
%----------------------------------------------------------------------------
\section{Alternatívák}
%----------------------------------------------------------------------------
Az imént vázolt igények kielégítésére rengeteg kiforrott megoldás létezik már, melyek jó példát mutattak a saját alkalmazásom fejlesztése során.

%----------------------------------------------------------------------------
\subsection{Grafana}
%----------------------------------------------------------------------------
A Grafana\cite{grafana} az egy nyílt forráskódú platformfüggetlen vizualizációs web alkalmazás.
Egy támogatott adatbázishoz csatlakoztatva különféle interaktív gráfokat és diagrammokat generál.
A testreszabhatóság maximalizásának érdekében különböző, akár harmadik fél által készített, bővítmények használatát is támogatja, 
melyekkel új adatforrások és panel típusok integrálhatók. 
A \ref{fig:grafana}-es ábra egy jó példa arra, hogy hogyan néz ki egy általános Grafana felület.

\begin{figure}[!ht]
    \centering
    \includegraphics[width=150mm, keepaspectratio]{figures/grafana.png}
    \caption{A Grafana demo oldalának, a \url{https://play.grafana.org}-nak a felülete}
    \label{fig:grafana}
\end{figure}
    
%----------------------------------------------------------------------------
\subsection{Kibana}
%----------------------------------------------------------------------------
A Kibana\cite{kibana} jelentősen hasonlít a Grafanához, azonban amíg a utóbbit inkább az időben változó metrikák vizualizálására használják például processzor leterheltség vagy memória használat,
addig az előbbit elsődlegesen az Elasticsearch\footenotemark adatok, főként napló bejegyzések, analizálására használják.

\footnotetext{Ingyenes és nyílt forráskódú index alapú keresőmotor}

\begin{figure}[!ht]
    \centering
    \includegraphics[width=150mm, keepaspectratio]{figures/kibana-dashboard.png}
    \caption{Egy példa a Kibana kezelőfelületére}
    \label{fig:kibana}
\end{figure}

%----------------------------------------------------------------------------
\subsection{Kubernetes Dashboard (Web UI)}
%----------------------------------------------------------------------------
A Kubernetes Dashboard\cite{kubernetes-dashboard} elsősorban nem a különböző adatok vizualizálását szolgálja, inkább a klaszter menedzselését próbálja egyszerűbbé és jobban áttekinthetővé tenni.
Azonban egy jó példa arra, hogy egy rendszer webes kezelőfelületének, milyennek is kell lennie.

\begin{figure}[!ht]
    \centering
    \includegraphics[width=150mm, keepaspectratio]{figures/kubernetes-dashboard.png}
    \caption{A Kubernetes Dashboard felülete}
    \label{fig:kibana}
\end{figure}