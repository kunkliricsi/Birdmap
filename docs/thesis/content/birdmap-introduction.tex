%----------------------------------------------------------------------------
\chapter{Tervek és alternatívák}
\label{chapt:birdmap-introduction}
%----------------------------------------------------------------------------
Ebben a fejezetben bemutatom a fejlesztés előtti állapotot, amikor még csak tervezgettük, hogy milyen is legyen az alkalmazás.
Illetve bemutatot, néhány vizualizációs alternatívát, melyeket ihletszerzés gyanánt használtam.

%----------------------------------------------------------------------------
\section{Tervezés}
%----------------------------------------------------------------------------
Az első dolgunk az volt, hogy Kristóffal és Marcellel beültünk egy Teams\footnotemark-en tartott gyűlésre,
ahol elmagyarázták nagyvonalakban, hogy hogyan is működik a rendszer, mik az egyes kompnensek feladatai. 
Ezek után az előttem álló fejlesztésre váró alkalmazás részleteit beszéltük meg, az elvárt igényeket azzal kapcsolatban.
Itt rögtön több ötlet is felmerült, melyek közül a legkiemelkedőbbek:
\begin{itemize}
    \item \textbf{Hőtérkép}. Hasznos lenne egy olyan felület, ahol az eszközök GPS koordinátái és a seregély detektálást jelző üzenetek alapján, meg lehetne jeleníteni a seregélyek hozzávetőleges előfordulásának helyeit és gyakoriságát egy térképen, hőtérképes formában.
    \item \textbf{Eszköz állapotok}. Jelenleg a Command and Control mikroszolgáltatás felé indított kéréseken kívül, nincs lehetőség a kihelyezett eszközök állapotának vizsgálatára. Szükség lenne egy olyan felületre, ahol ezek állapotai láthatóak, esetleg dinamikusan is frissülnek.
    \item \textbf{Diagrammok}. A hőtérképen kívül egyéb olyan diagrammok is hasznosak lehetnek, ahol látható például, hogy melyik eszköz melyik percben észlelt madárhangot vagy, hogy egy eszköz összesen hány madárhangot észelt. Minnél több információ, annál jobb.
\end{itemize}

\footnotetext{Microsoft Teams: Csevegő és gyülekezés tartó alkalmazás.}
%----------------------------------------------------------------------------
\section{Alternatívák}
%----------------------------------------------------------------------------
Az imént vázolt igények kielégítésére rengeteg kiforrott megoldás létezik már. Ezek közül bemutatok néhányat, melyek jó útmutatást adtak az alkalmazásom fejlesztése során

\subsection{Grafana}
