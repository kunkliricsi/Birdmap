%----------------------------------------------------------------------------
\chapter{A Birdnetes részletes bemutatása}
\label{chapt:birdnetes-introduction}
%----------------------------------------------------------------------------
Ebben a fejezetben ismertetem a Birdnetes mikroszolgáltatás rendszerének architektúráját.
Részletesen kifejtem az alkalmazásom szempontjából fontos komponensek feladatát és működését.
Majd egy példával ábrázolom a rendszer hangfelismerő folyamatát.

%----------------------------------------------------------------------------
\section{Gyors elméleti összefoglaló}
%----------------------------------------------------------------------------
Ez a szakasz nem azt a célt szolgálja, hogy minnél részletesebb képet mutasson az itt leírt technológiákról.
Ez csupán egy rövid összefoglaló a Birdnetes működésének megértése szempontjából elengedhetetlen technológiákról és elvekről,
hogy valamilyen szinten tisztában legyünk a fejezetben elhangzó kifejezésekkel.

%----------------------------------------------------------------------------
\subsection{Cloud, felhő}
%----------------------------------------------------------------------------
A cloud lényegében annyit jelent, hogy a szervert, amin az alkalmazás fut, nem a fejlesztőnek kell üzemeltetnie,
hanem valamilyen másik szervezet\footnotemark által vannak karban tartva. 
Ez több okból is hasznos:
\begin{itemize}
    \item Olcsóbb. Nem kell berendezéseket vásárolni, nincs üzemeltetési díj. Az egyetlen költség a bérlés, ami általában töredéke annak, amit akkor fizetnénk ha magunk csinálnánk az egészet.
    \item Gyorsabb fejlesztés. Az alkalmazás futtatására használt szervereket általában a fejlesztő nem látja, ezekkel nem kell foglalkoznia. Ha az alkalmazásnak hirtelen nagyobb erőforrás igénye lesz, a rendszer automatikusan skálázódik.
    \item Nagyobb megbízhatóság. Az ilyen szolgáltatást nyújtó szervezeteknek ez az egyik legnagyobb feladata. Az alkalmazás bárhol és bármikor elérhető.
\end{itemize}

\footnotetext{Ilyenek például a Microsoft Azure, az Amazon Web Services vagy a Google Cloud.}
%----------------------------------------------------------------------------
\subsubsection{Mikroszolgáltatások}
%----------------------------------------------------------------------------
A mikroszolgáltatások nem sok mindenben különböznek egy általános szolgáltatástól. 
Ugyan úgy valamilyen kéréseket kiszolgáló egységek, legyen az web kérések kiszolgálása HTTP-n keresztül
vagy akár parancssori utasítások feldolgozása. Az egyetlen fő különbség az a szolgáltatások felelősségköre.
A mikroszolgáltatások fejlesztésénél a fejlesztők elsősorban arra törekednek, hogy egy komponensnek minnél kevesebb feladata és függősége legyen, 
ezzel megnő a tesztelhetőség és könyebb a skálázhatóság.

%----------------------------------------------------------------------------
\subsubsection{Konténerek}
%----------------------------------------------------------------------------
A konténer technikailag semmivel sem több mint egy Linux-on futó processz amelyre különböző korlátozásokat szabtak.
Ilyen korlátozások lehetnek például, hogy a konténer nem látja a teljes fájlrendszert, annak csak egy kijelölt részét,
megadható a konténer által használható processzor és memória igény vagy akár korlátozható az is, hogy a konténer hogyan használhatja a hálózatot.
Léteznek eszközök, például a Docker\footnote{https://www.docker.com/}, mely lehetővé teszi a fejlesztők számára az ilyen konténerek könnyed létrehozását és futtatását.

%----------------------------------------------------------------------------
\subsubsection{Kubernetes}
%----------------------------------------------------------------------------
A Kubernetes\footnote{https://kubernetes.io/} az ilyen komplex konténerizált mikroszolgáltatás rendszerek menedzselésének könnyítését szolgálja.
Kihasználja és ötvözi az imént említett technológiák előnyeit, hogy egy robosztus rendszert alkosson. 
Használatával felgyorsulhat és automatizált lehet az egyes konténerek telepítése, futtatása, de talán a legfőbb előnye,
hogy segítségével könnyedén megoldható a rendszert ért terhelési igények szerinti dinamikus skálázódás. 
Azok a mikroszolgáltatások, amikre a rendszernek épp nincs szüksége, nem futnak, nem igényelnek erőforrást a szerveren,
így nem kell utánnuk fizetni sem. Ezzel ellentétben, ha valamely szolgáltatás után hirtelen megnő az igény,
akkor az könnyedén duplikálható.

%----------------------------------------------------------------------------
\subsection{MQTT}
%----------------------------------------------------------------------------
Az MQTT (Message Queue Telemetry Transport) az egy kliens-szerver publish/subscribe üzenetküldő protokoll. Könnyű implementálni és alacsony a sávszélesség igénye,
mellyel tökéletes jelöltje a Machine to Machine (M2M), illetve az Internet of Things (IoT) kommunikáció megvalósítására.
Működéséhez szükség van egy szerverre, amelynek feladata a beérkező üzenetek továbbküldése témák alapján. Egyes kliensek fel tudnak iratkozni bizonyos témákra, míg más kliensek publikálnak
és a szerver levezényli a két fél között a kommunikációt.

%----------------------------------------------------------------------------
\subsection{Open API}
%----------------------------------------------------------------------------
Az Open API egy nyilvános alkalmazás-programozási leíró, amely a fejlesztők számára hozzáférést biztosít egy másik alkalmazáshoz.
Az API-k lírják és meghatározzák, hogy egy alkalmazás hogyan kommunikálhat egy másikkal, 
melyet használva a fejlesztők könnyedén képesek a kommunikációra képes kódot írni vagy generálni.

%----------------------------------------------------------------------------
\section{Rendszerszintű architektúra}
%----------------------------------------------------------------------------
A Birdnetes fejlesztése során kifejezetten fontos szerepe volt a mikroszolgáltatás alapú rendszerek elvei követésének.
A rendszer egy Kubernetes klaszterben van telepítve és több kisebb komponensből áll, melyek egymás között a HTTP és az MQTT protokollok segítségével kommunikálnak.
A rendszer összes szolgáltatásának van egy Open API leírója, melyet használva hamar volt egy olyan kódbázisom, amely képes volt a rendszerrel való kommunikációra.

%----------------------------------------------------------------------------
\subsection{Főbb komponensek}
%----------------------------------------------------------------------------
A \ref{fig:birdnetes-components}-es ábrán láthatóak a rendszer komponensei, melyek mind egy-egy mikroszolgáltatás.
Az egymás mellett lévő kék levélborítékok az MQTT kommunikációt jelölik,
amellyel például a természetben elhelyezett eszközök felé irányuló kommunikációja is történik.
A következő alszakaszokban bemutatom az alkalmazásom szempontjából fontosabb komponenseket.

\begin{figure}[!ht]
\centering
\includegraphics[width=150mm, keepaspectratio]{figures/architecture-redesigned.png}
\caption{A Birdnetes rendszer architektúrája}
\label{fig:birdnetes-components}
\end{figure}

%----------------------------------------------------------------------------
\subsubsection{Input Service}
%----------------------------------------------------------------------------
A kihelyezett IoT eszközök által felvett hangfájlok ezen a komponensen keresztül érkeznek be a rendszerbe.
Itt történik a hanganyaghoz tartozó metaadatok lementése az Input Service saját adatbázisába. 
Ilyenek például a beküldő eszköz azonosítója, a beérkezés dátuma vagy a hangüzenet rendszerszintű egyedi azonosítója.
Amint a szolgáltatás a berékezett üzenettel kapcsolatban elvégezte az összes feladatát, 
publikál egy üzenetet az MQTT üzenetsorra a többi kliensnek feldolgozásra.

%----------------------------------------------------------------------------
\subsubsection{AI Service}
%----------------------------------------------------------------------------
Az AI Service példányai fogadják az Input Service-től érkező üzeneteket és elkezdik klasszifikálni az abban található hanganyagot.
Meghatározzák, hogy a hanganyag mekkora valószínűséggel volt seregély hang vagy sem.
Ennek eredményét a hangminta egyedi azonosítójával együtt publikálják egy másik üzenetsoron.

%----------------------------------------------------------------------------
\subsubsection{Guard Service}
%----------------------------------------------------------------------------
A Guard Service feliratkozik az AI Service által publikált üzenetek témájára 
és valamilyen valószínűségi kritérium alapján eldönti, hogy a hangminta tartalmaz-e seregély hangot.
Ha igen, akkor az üzenetsoron küld egy riasztás parancsot a hanganyagot küldő eszköznek.

%----------------------------------------------------------------------------
\subsubsection{Command and Control Service}
%----------------------------------------------------------------------------
A Command and Control Service az előzőekkel ellentétben nem vesz részt a minták fogadásában, feldolgozásában vagy kezelésében.
Felelősége az eszközök és azok szenzorai állapotának menedzselése és követése.
Ezen keresztül lehet az egyes eszközöket ki- és bekapcsolni.