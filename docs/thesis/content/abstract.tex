\pagenumbering{roman}
\setcounter{page}{1}

\selecthungarian

%----------------------------------------------------------------------------
% Abstract in Hungarian
%----------------------------------------------------------------------------
\chapter*{Kivonat}\addcontentsline{toc}{chapter}{Kivonat}

Adott egy tanszéken fejlesztett felhő alapú elosztott rendszer, melynek eszközei madárhangok azonosítására képesek.
Ha a rendszer úgy észleli, hogy az egyik álatala vezérelt eszköz mikrofonja felvételén madárhang található,
akkor riasztást kezdeményez az eszközön ezzel elijesztve a madarat ezáltal megóvva a növényzetet.

A rendszernek több kisebb komponense van, amelyek rengeteg adatot dolgoznak fel és nincs jelenleg egy olyan egységes grafikus felület ahol a rendszer teljes állapotát
át lehetne tekinteni, illetve ahol a feldolgozott adatokat vizualizálni lehetne.

A piacon létezik már több olyan szoftver csomag, amely hasonló problémákra próbál megoldást nyújtani, de ezek sem mindig
tudják kielégíteni azokat a speciális igényeket, amelyek egy ilyen rendszernél felmerülnek.

Jelen szakdolgozat célja egy olyan vizualizációs megoldás bemutatása, amelynek segítségével a rendszer könnyedén áttekinthető
és kezelhető. A tanszéki rendszer által kezelt eszközök a felületen is vezérelhetők 
és azok működéséről különböző statisztikákat felhasználva egyszerűen értelmezhető diagrammok generálódnak.

A backend megvalósítására az ASP.NET Core-t választottam, mely platformfüggetlen megoldást nyújt a web kérések kiszolgálására.
A frontend-et a React.js használatával készítettem, mely segítségével egyszerűen és gyorsan lehet reszponzív felhasználói felületeket készíteni.
Dolgozatomban bemutatom a tanszéken fejlesztett rendszert, a mikroszolgáltatások vizualizálásának alternatíváit,
ismertetem az általam választott technológiákat és a készített alkalmazás felépítését.

\vfill
\selectenglish


%----------------------------------------------------------------------------
% Abstract in English
%----------------------------------------------------------------------------
\chapter*{Abstract}\addcontentsline{toc}{chapter}{Abstract}

There is a department developed cloud-based distributed system whose devices are capable of identifying bird sounds.
If the system detects a bird's voice on the recording of a microphone on one of the devices, it will trigger
an alarm on the device scaring the bird away thereby protecting the vegetation.

The system has several smaller components that process a lot of data and currently there is no unified graphical user interface where the overall state of the system 
could be reviewed or where the processed data could be visualized.

The purpose of this thesis is to present a visualization solution that allows the users to easily review
and manage the system. The devices maintained by the department developed system can be controlled on the interface
and easy-to-understand diagrams are generated using statistics about their operation.

I chose ASP.NET Core as the framework for the backend, which provides a platform-independent solution for serving web requests.
The frontend was created using React.js, which allows for an easy and quick way to create responsive user interfaces.
In my thesis I present the system developed at the department, the alternatives of visualization of microservices,
I describe the technologies I have chosen and the structure of the application I have created.

\vfill
\selectthesislanguage

\newcounter{romanPage}
\setcounter{romanPage}{\value{page}}
\stepcounter{romanPage}