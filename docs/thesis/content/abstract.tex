\pagenumbering{roman}
\setcounter{page}{1}

\selecthungarian

%----------------------------------------------------------------------------
% Abstract in Hungarian
%----------------------------------------------------------------------------
\chapter*{Kivonat}\addcontentsline{toc}{chapter}{Kivonat}

Napjainkban a mezőgazdaságban egyre elterjedtebbek a dolgok internetére (Internet of Things
– IoT) épülő megoldások, ezek viszont nagy mennyiségű adatot generálnak, amelyek
feldolgozása tradicionális rendszerekkel nehézkes. Erre a problémára próbál megoldást
nyújtani egy, a tanszéken kifejlesztett felhő-natív adatfeldolgozó és elemző rendszer.

A rendszer használata közben a különböző komponensek változó méretű és jellegű
terhelésnek vannak kitéve, ami nehézkessé teszi a rendszer működésének áttekintését, a
folyamat vizualizálását. 

Jelen szakdolgozat célja egy olyan vizualizációs megoldás bemutatása, amelynek segítségével a rendszer könnyedén áttekinthető
és kezelhető. A tanszéki rendszer által kezelt eszközök a felületen is vezérelhetők 
és azok működéséről különböző statisztikákat felhasználva egyszerűen értelmezhető diagrammok generálódnak.

A backend megvalósítására az ASP.NET Core-t választottam, mely platformfüggetlen megoldást nyújt a web kérések kiszolgálására.
A frontend-et a React.js használatával készítettem, mely segítségével egyszerűen és gyorsan lehet reszponzív felhasználó felületeket készíteni.
Dolgozatomban bemutatot a tanszéken fejlesztett rendszert, a mikroszolgáltatások vizualizálásának alternatíváit,
ismertetem az általam választott technológiákat és a készített alkalmazás felépítését.

\vfill
\selectenglish


%----------------------------------------------------------------------------
% Abstract in English
%----------------------------------------------------------------------------
\chapter*{Abstract}\addcontentsline{toc}{chapter}{Abstract}

Nowadays, the internet of things is becoming more and more prevalent for IT systems in the agriculture.
These generate large amounts of data the processing of which is rather cumbersome with traditional systems. 
A department developed system is trying to solve this problem to providing a cloud-native data processing and analysis solution.

While using the system, the various components are subjected to loads varying in size and nature
which makes it difficult to review the operation of the system and visualize the process.

The purpose of this thesis is to present a visualization solution that allows the users to easily review
and manage the system. The devices maintained by the department developed system can be controlled on the interface
and easy-to-understand diagrams are generated using statistics about their operation.

I chose the ASP.NET Core to implement the backend, which provides a platform-independent solution for serving web requests.
The frontend was created using React.js, which allows for an easy and quick way to create responsive user interfaces.
In my thesis I present the system developed at the department, the alternatives of visualization of microservices,
I describe the technologies I have chosen and the structure of the application I have created.

\vfill
\selectthesislanguage

\newcounter{romanPage}
\setcounter{romanPage}{\value{page}}
\stepcounter{romanPage}